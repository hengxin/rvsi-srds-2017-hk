\section{Definition of RVSI}

% si
%%%%%%%%%%%%%%%
\begin{frame}{}
  $\text{Transaction } T_i:\quad s_i\quad (r_i/w_i)^{+}\quad c_i/a_i$

  \begin{description}
    \item[$s_i$:] \texttt{start} operation
    \item[$r_i/w_i$:] \texttt{read/write} operation
    \item[$c_i/a_i$:] \texttt{commit/abort} operation
  \end{description}

  \pause
  \vspace{0.30cm}
  \begin{description}[$w_i(x_i)$:]
    \item[$x_i$:] version $i$ of data item $x$ written by $T_i$
    \item[$r_i(x_j)$:] transaction $T_i$ reading $x_j$
    \item[$w_i(x_i)$:] transaction $T_i$ writing $x_i$
  \end{description}
\end{frame}
%%%%%%%%%%%%%%%

%%%%%%%%%%%%%%%
\begin{frame}{}
  History: modelling an execution of a transactional key-value store
  \begin{itemize}
    \item \emph{time-precedes partial order} $\timeprech$ over operations
  \end{itemize}

  \pause
  \vspace{0.50cm}
  \begin{columns}
    \column{0.50\textwidth}
      Two transactions are \emph{concurrent} if
      \[
	s_i \timeprech c_j \land s_j \timeprech c_i
      \]
    \column{0.50\textwidth}
      \fignocaption{width = 0.60\textwidth}{figs/concur-tx}
  \end{columns}
\end{frame}
%%%%%%%%%%%%%%%

%%%%%%%%%%%%%%%
% \begin{frame}{}
%   Snapshot isolation requires that
%   \vspace{10pt}
%   \begin{description}[Snapshot Write:]
%     \item[Snapshot Read:] Each transaction reads data from the ``lastest'' snapshot as of the time the transaction started
%     \item[Snapshot Write:] No write-conflicting concurrent transactions
%   \end{description}
% \end{frame}
%%%%%%%%%%%%%%%

%%%%%%%%%%%%%%%
\begin{frame}{}
  A history $h$ is in \emph{\large snapshot isolation} iff it satisfies \citeinbeamer{Adya}{Thesis}{99} \\[0.20cm]
  \begin{description}[Snapshot Write:]
    \item[Snapshot Read:] All reads of transaction $T_i$ occur at $T_i$'s start time.%
      \uncover<2->{%
	\begin{align*}
	  \forall &r_i(x_{j \neq i}), w_{k \neq j}(x_k), c_k \in h: \\
	  & (c_j \in h \land c_j \timeprech s_i)
	   \land (s_i \timeprech c_k \lor c_k \timeprech c_j).
	\end{align*}%
      }%
    \item[Snapshot Write:] No concurrent committed transactions may write the same data item. (WCF: write-conflict freedom)
      \uncover<2->{
	\begin{align*}
	  \forall w_i(x_i), w_{j \neq i}(x_j) \in h \implies (c_i \timeprech s_j \lor c_j \timeprech s_i).
	\end{align*}
      }
  \end{description}
\end{frame}
%%%%%%%%%%%%%%%


%%%%%%%%%%%%%%%
\begin{frame}{}
  Principles of \rvsi{}:
  \begin{itemize}
    \setlength{\itemsep}{10pt}
    \item<1-> Using parameters ($k_1, k_2, k_3$) to control the severity of the anomalies w.r.t SI
    \item<2-> $\rc{}~\footnotemark \supset \rvsi(k_1, k_2, k_3) \supset \si{}$
    \item<2-> $\rvsi(\infty,\infty,\infty) = \rc \qquad \rvsi(1,0,\ast) = \si$
  \end{itemize}
  \footnotetext{RC: Read Committed isolation.}
\end{frame}
%%%%%%%%%%%%%%%

%%%%%%%%%%%%%%%
\begin{frame}{}
  \begin{quote}
    Each transaction reads data from the ``latest'' snapshot as of the time the transaction started.\\
    \hfill -- The ``Snapshot Read'' property of SI
  \end{quote}

  \vspace{0.60cm}
  \rvsi{} relaxes ``Snapshot Read'' in three ways:
  \pause
  \begin{description}[$\kthreesv$ (Snapshot View):]
    \setlength{\itemsep}{3pt}
    \item[$\konebv$ (Backward View):] ``stale'' data versions		\hfill \red{staleness $\le k_1$}
      \pause
    \item[$\ktwofv$ (Forward View):] ``concurrent'' data versions  	\hfill \red{concurrency level $\le k_2$}
      \pause
    \item[$\kthreesv$ (Snapshot View):] ``non-snapshot'' data versions	\hfill \red{distance $\le k_3$}
  \end{description}
\end{frame}
%%%%%%%%%%%%%%%

%%%%%%%%%%%%%%%
\begin{frame}{}
  % \fignocaption{width = 1.0\textwidth}{figs/rvsi-definition.pdf}
  % \resizebox{1.00\textwidth}{!}{\input{tikz/rvsi-def}}
  \scalebox{0.30}{\input{tikz/rvsi-def}}
\end{frame}
%%%%%%%%%%%%%%%

%%%%%%%%%%%%%%%
\begin{frame}{}
  \begin{align*}
    & \red{(\konebv{})} \\
    \forall r_i(x_{j}), w_k(x_k), c_k \in h: 
    & \left(c_j \in h \land \bigwedge_{k=1}^{m} \left(c_j \hp c_k \hp s_i \right)\right) \Rightarrow m < k_1, \\[10pt]
    & \red{(\ktwofv{})} \\
    \forall r_i(x_{j}), w_k(x_k), c_k \in h:
    & \left(c_j \in h \land \bigwedge_{k=1}^{m} \left(s_i \hp c_k \hp c_j \right)\right) \Rightarrow m \le k_2, \\[10pt]
    & \red{(\kthreesv{})} \\
    \forall r_i(x_j), r_i(y_l), w_k(x_k), c_k \in h:
    & \left(\bigwedge_{k=1}^{m} \left(c_j \hp c_k \hp c_l \right)\right) \Rightarrow m \le k_3.
  \end{align*}
\end{frame}
%%%%%%%%%%%%%%%

%%%%%%%%%%%%%%%
\begin{frame}{}
  \begin{equation*}
    h \in \rvsi{} \iff h \in \konebv{} \cap \ktwofv{} \cap \kthreesv{} \cap \wcf{}
  \end{equation*}

  \vspace{0.50cm}
  \[
    \rvsi(\infty,\infty,\infty) = \rc \qquad \rvsi(1,0,\ast) = \si
  \]
\end{frame}
%%%%%%%%%%%%%%%
