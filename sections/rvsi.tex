\section{Definition of RVSI}

% % si
%%%%%%%%%%%%%%%
\begin{frame}{}
    Transaction $T_i$:
    \begin{itemize}
      \item begins with a \texttt{start} operation $s_i$ 
      \item contains a sequence of \texttt{read} or \texttt{write} operations
      \item ends with a \texttt{commit} operation $c_i$ or an \texttt{abort} operation $a_i$
    \end{itemize}

    \pause
    \vspace{0.30cm}
    History: modelling an execution of a transactional key-value store
    \begin{itemize}
      \item $w_i(x_i)$: transaction $T_i$ writing version $i$ of data item $x$
      \item $r_i(x_j)$: transaction $T_i$ reading version $j$ of data item $x$ written by $T_j$
      \item \emph{time-precedes partial order} $\timeprech$ over operations
      \item $T_i$ and $T_j$ are concurrent:
	\[
	  s_i \timeprech c_j \land s_j \timeprech c_i
	\]
    \end{itemize}
\end{frame}
%%%%%%%%%%%%%%%

%%%%%%%%%%%%%%%
\begin{frame}{}
  Snapshot isolation requires that:
  \begin{description}[Snapshot Write:]
    \item[Snapshot Read:] Each transaction read data from the ``lastest'' snapshot as of the time the transaction started.
    \item[Snapshot Write:] No \red{write}-conflicting concurrent transactions
  \end{description}
\end{frame}
%%%%%%%%%%%%%%%

%%%%%%%%%%%%%%%
\begin{frame}{}
  A history $h$ is in snapshot isolation if and only if it satisfies \citeinbeamer{Adya}{Thesis}{99} \\[0.20cm]
  \begin{description}[Snapshot Write:]
    \item[Snapshot Read:] All reads of transaction $T_i$ occur at $T_i$'s start time.
      \begin{align*}
	\forall &r_i(x_{j \neq i}), w_{k \neq j}(x_k), c_k \in h: \\
	& (c_j \in h \land c_j \timeprech s_i)
	 \land (s_i \timeprech c_k \lor c_k \timeprech c_j).
      \end{align*}
    \item[Snapshot Write:] No concurrent committed transactions may write the same data item.
      \begin{align*}
	\forall w_i(x_i), w_{j \neq i}(x_j) \in h \implies (c_i \timeprech s_j \lor c_j \timeprech s_i).
      \end{align*}
  \end{description}
\end{frame}
%%%%%%%%%%%%%%%

%%%%%%%%%%%%%%%
\begin{frame}{}
\end{frame}
%%%%%%%%%%%%%%%

%%%%%%%%%%%%%%%
\begin{frame}{}
\end{frame}
%%%%%%%%%%%%%%%

%%%%%%%%%%%%%%%
\begin{frame}{}
\end{frame}
%%%%%%%%%%%%%%%

%%%%%%%%%%%%%%%
\begin{frame}{}
\end{frame}
%%%%%%%%%%%%%%%

%%%%%%%%%%%%%%%
\begin{frame}{}
\end{frame}
%%%%%%%%%%%%%%%


%%%%%%%%%%%%%%%
\begin{frame}{Principle of \rvsi{}}
  \begin{itemize}
    \setlength{\itemsep}{10pt}
    \item<1-> Parameters ($k_1, k_2, k_3$) to control the severity of the anomalies w.r.t SI
    \item<2-> $\rc{}~\footnotemark \supset \rvsi(k_1, k_2, k_3) \supset \si{}$
    \item<2-> $\rvsi(\infty,\infty,\infty) = \rc \qquad \rvsi(1,0,\ast) = \si$
  \end{itemize}
  \footnotetext{RC: Read Committed}
\end{frame}
%%%%%%%%%%%%%%%

%%%%%%%%%%%%%%%
\begin{frame}{Principle of \rvsi{}}
  \begin{quote}
    $\ldots$\\
    \hfill -- ``Snapshot Read'' property of SI
  \end{quote}

  \begin{enumerate}[<+->]
    \setlength{\itemsep}{10pt}
    \item ``stale'' data versions		\hfill \uncover<4->{\red{bounded staleness}}
    \item ``concurrent'' data versions  	\hfill \uncover<5->{\red{bounded concurrency level}}
    \item ``non-snapshot'' data versions	\hfill \uncover<6->{\red{bounded distance}}
  \end{enumerate}
\end{frame}
%%%%%%%%%%%%%%%

%%%%%%%%%%%%%%%
\begin{frame}{Illustration of RVSI}
  \fignocaption{width = 0.85\textwidth}{figs/rvsi-definition.pdf}
\end{frame}
%%%%%%%%%%%%%%%

%%%%%%%%%%%%%%%
\begin{frame}{Definition of RVSI}
  \begin{align*}
    & \red{(\konebv{})} \\
    \forall r_i(x_{j}), w_k(x_k), c_k \in h: 
    & \left(c_j \in h \land \bigwedge_{k=1}^{m} \left(c_j \hp c_k \hp s_i \right)\right) \Rightarrow m < k_1, \\[10pt]
    & \red{(\ktwofv{})} \\
    \forall r_i(x_{j}), w_k(x_k), c_k \in h:
    & \left(c_j \in h \land \bigwedge_{k=1}^{m} \left(s_i \hp c_k \hp c_j \right)\right) \Rightarrow m \le k_2, \\[10pt]
    & \red{(\kthreesv{})} \\
    \forall r_i(x_j), r_i(y_l), w_k(x_k), c_k \in h:
    & \left(\bigwedge_{k=1}^{m} \left(c_j \hp c_k \hp c_l \right)\right) \Rightarrow m \le k_3.
  \end{align*}
\end{frame}
%%%%%%%%%%%%%%%

%%%%%%%%%%%%%%%
\begin{frame}{Definition of \rvsi{}}
  \begin{equation*}
    h \in \rvsi{} \iff h \in \konebv{} \cap \ktwofv{} \cap \kthreesv{} \cap \wcf{}.
  \end{equation*}

  \vspace{0.50cm}
  \begin{theorem}
    \[
      \emph{\rvsi}(1,0,\ast) = \emph{\si}.
    \]
  \end{theorem}
\end{frame}
%%%%%%%%%%%%%%%

%%%%%%%%%%%%%%%
\begin{frame}{}
\end{frame}
%%%%%%%%%%%%%%%

%%%%%%%%%%%%%%%
\begin{frame}{}
\end{frame}
%%%%%%%%%%%%%%%
