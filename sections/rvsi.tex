\section{Definition of RVSI}

% si
%%%%%%%%%%%%%%%
\begin{frame}{}
    Transaction $T_i$:
    \begin{itemize}
      \item begins with a \texttt{start} operation $s_i$ 
      \item contains a sequence of \texttt{read} or \texttt{write} operations
      \item ends with a \texttt{commit} operation $c_i$ or an \texttt{abort} operation $a_i$
    \end{itemize}

    \pause
    \vspace{0.30cm}
    History: modelling an execution of a transactional key-value store
    \begin{itemize}
      \item $w_i(x_i)$: transaction $T_i$ writing version $i$ of data item $x$
      \item $r_i(x_j)$: transaction $T_i$ reading version $j$ of data item $x$ written by $T_j$
      \item \emph{time-precedes partial order} $\timeprech$ over operations
      \item $T_i$ and $T_j$ are concurrent:
	\[
	  s_i \timeprech c_j \land s_j \timeprech c_i
	\]
    \end{itemize}
\end{frame}
%%%%%%%%%%%%%%%

%%%%%%%%%%%%%%%
\begin{frame}{}
  Snapshot isolation requires that:
  \begin{description}[Snapshot Write:]
    \item[Snapshot Read:] Each transaction read data from the ``lastest'' snapshot as of the time the transaction started.
    \item[Snapshot Write:] No \red{write}-conflicting concurrent transactions
  \end{description}
\end{frame}
%%%%%%%%%%%%%%%

%%%%%%%%%%%%%%%
\begin{frame}{}
  A history $h$ is in snapshot isolation if and only if it satisfies \citeinbeamer{Adya}{Thesis}{99} \\[0.20cm]
  \begin{description}[Snapshot Write:]
    \item[Snapshot Read:] All reads of transaction $T_i$ occur at $T_i$'s start time.
      \begin{align*}
	\forall &r_i(x_{j \neq i}), w_{k \neq j}(x_k), c_k \in h: \\
	& (c_j \in h \land c_j \timeprech s_i)
	 \land (s_i \timeprech c_k \lor c_k \timeprech c_j).
      \end{align*}
    \item[Snapshot Write:] No concurrent committed transactions may write the same data item.
      \begin{align*}
	\forall w_i(x_i), w_{j \neq i}(x_j) \in h \implies (c_i \timeprech s_j \lor c_j \timeprech s_i).
      \end{align*}
  \end{description}
\end{frame}
%%%%%%%%%%%%%%%

%%%%%%%%%%%%%%%
\begin{frame}{}
\end{frame}
%%%%%%%%%%%%%%%

%%%%%%%%%%%%%%%
\begin{frame}{}
\end{frame}
%%%%%%%%%%%%%%%

%%%%%%%%%%%%%%%
\begin{frame}{}
\end{frame}
%%%%%%%%%%%%%%%

%%%%%%%%%%%%%%%
\begin{frame}{}
\end{frame}
%%%%%%%%%%%%%%%

%%%%%%%%%%%%%%%
\begin{frame}{}
\end{frame}
%%%%%%%%%%%%%%%


%%%%%%%%%%%%%%%
\begin{frame}{}
  Principles of \rvsi{}:
  \begin{itemize}
    \setlength{\itemsep}{10pt}
    \item<1-> Using parameters ($k_1, k_2, k_3$) to control the severity of the anomalies w.r.t SI
    \item<2-> $\rc{}~\footnotemark \supset \rvsi(k_1, k_2, k_3) \supset \si{}$
    \item<2-> $\rvsi(\infty,\infty,\infty) = \rc \qquad \rvsi(1,0,\ast) = \si$
  \end{itemize}
  \footnotetext{RC: Read Committed isolation.}
\end{frame}
%%%%%%%%%%%%%%%

%%%%%%%%%%%%%%%
\begin{frame}{}
  \begin{quote}
    Each transaction reads data from the ``latest'' snapshot as of the time the transaction started.\\
    \hfill -- The ``Snapshot Read'' property of SI
  \end{quote}

  \vspace{0.60cm}
  \rvsi{} relaxes ``Snapshot Read'' in three ways:
  \pause
  \begin{description}%[$\konebv$ (Backward View):]
    \setlength{\itemsep}{3pt}
    \item[$\konebv$ (Backward View):] ``stale'' data versions		\hfill \red{staleness $\le k_1$}
      \pause
    \item[$\ktwofv$ (Forward View):] ``concurrent'' data versions  	\hfill \red{forward level $\le k_2$}
      \pause
    \item[$\kthreesv$ (Snapshot View):] ``non-snapshot'' data versions	\hfill \red{distance $\le k_3$}
  \end{description}
\end{frame}
%%%%%%%%%%%%%%%

%%%%%%%%%%%%%%%
\begin{frame}{}
  % \fignocaption{width = 1.0\textwidth}{figs/rvsi-definition.pdf}
  \resizebox{1.00\textwidth}{!}{\tikzset{trans-node/.style = {draw, line width = 3pt, inner sep = 12pt}}
\tikzset{read-from/.style = {>={Stealth[length = 12pt]}, ->, dash pattern = on 20pt off 10pt, thick}}
\tikzset{knode/.style = {inner sep = 8pt, outer sep = 5pt, font=\fontsize{40}{40}\selectfont}}
\tikzset{toarrow/.style = {>={Stealth[length = 12pt]}, ->, ultra thick}}
\tikzset{dashline/.style = {ultra thick, dash pattern = on 20pt off 10pt}}
\tikzset{cs/.style = {outer sep = 5pt, font=\fontsize{40}{40}\selectfont}}  % c: commit; s: start

% #1: numbering
\newcommand{\xtrans}[1]{\node (x#1) [trans-node, on chain = x, font=\fontsize{30}{30}\selectfont] {$T_{x_#1}: w_{x_#1}(x_#1)$}}
\newcommand{\ytrans}[1]{\node (y#1) [trans-node, on chain = y, font=\fontsize{30}{30}\selectfont] {$T_{y_#1}: w_{y_#1}(y_#1)$}}

\begin{tikzpicture}[start chain = x going right,
  		start chain = y going right,
	      	node distance = 1.0cm,
	        font = \huge]
  % transactions updating x
  \begin{scope}
    \foreach \i in {1, ..., 6}
      \xtrans{\i};
  \end{scope}

  % transactions updating y
  \begin{scope}
    \node (y1) [trans-node, on chain = y, below left = 3.0cm and -2.0cm of x3, font=\fontsize{30}{30}\selectfont] {$T_{y_1}: w_{y_1}(y_1)$};
    \foreach \i in {2, ..., 6}
      \ytrans{\i};
  \end{scope}
   
  % transaction T_i reading x and y
  \begin{scope}
    \node (ti) [trans-node, below right = 2.0cm and 2.5cm of y2, font=\fontsize{35}{35}\selectfont] 
	  {$T_i: \hskip 3em r_i(x_2) \hskip 8em r_i(y_5) \hskip 4em$};
  \end{scope}

  \uncover<2->{
    % T_i reads x2
    \begin{pgfonlayer}{background}
	  \coordinate (c-rx2) at ($(ti.north) !0.5! (ti.north west)$);
	  \draw[read-from]  (x2.south) to (c-rx2); % [out = -70, in = 120] (c-rx2);
    \end{pgfonlayer}
    
    % T_i reads y5
    \coordinate (c-ry5) at ($(ti.north) !0.5! (ti.north east)$);
    \draw[read-from] (y5.south) to (c-ry5);
  }

  \uncover<3->{
    % k1
    \def\konespace{11cm}
    \begin{scope}[on background layer]
	% c-x2-top: commit time of T_x2
	\draw let \p{p-x2} = (x2.east), \p{p-ti} = (ti.north west) in
	    node (c-x2-top) [cs] at ($(\x{p-x2}, \y{p-ti}) + (0.0cm, \konespace)$) {$c_{x_2}$};
	\draw[dashline] (x2.north east) to (c-x2-top.south);

	% s-ti-top: start time of T_i (drawn at the top)
	\node (s-ti-top) [cs] at ($(ti.north west) + (0.0cm, \konespace)$) {$s_i$};
	\draw[dashline] let \p{p-ti} = (ti.west), \p{p-x4} = (x4.west) in
		  (s-ti-top.south) to (\x{p-ti}, \y{p-x4});

	% k1 at center of c-x2-top and s-ti-top
	\node (k1) [knode] at ($(c-x2-top) !0.5! (s-ti-top)$) {$k_1$};
	\draw [toarrow] (k1) to (c-x2-top);
	\draw [toarrow] (k1) to (s-ti-top);
    \end{scope}
  }

  \uncover<4->{
    % k2
    \def\ktwospace{1.5cm}
    \begin{scope}[on background layer]
	% s-ti-bot: start time of T_i (drawn at the bottom)
	\node (s-ti-bot) [cs] at ($(ti.south west) - (0.0cm, \ktwospace)$) {$s_i$};
	\draw[dashline] let \p{p-ti} = (ti.west), \p{p-x4} = (x4.west) in
	(s-ti-bot.north) to (\x{p-ti}, \y{p-x4});

	% c-y5-bot: commit time of T_y5
	\draw let \p{p-y5} = (y5.east), \p{p-ti} = (ti.south west) in
	node (c-y5-bot) [cs] at ($(\x{p-y5}, \y{p-ti}) - (0.0cm, \ktwospace)$) {$c_{y_5}$};
	\draw [dashline] (y5.south east) to (c-y5-bot.north);

	% k2 at center of s-ti-bot and c-y5-bot
	\node (k2) [knode] at ($(s-ti-bot) !0.5! (c-y5-bot)$) {$k_2$};
	\draw [toarrow] (k2) to (s-ti-bot);
	\draw [toarrow] (k2) to (c-y5-bot);
    \end{scope}
  }

  \uncover<5->{
    % k3
    \begin{scope}[on background layer]
	% (invisible; as positioning anchor) s-ti-mid: start time of T_i (drawn in the middle)
	\node (s-ti-mid) [cs] at ($(s-ti-top) !0.35! (s-ti-bot)$) {};

	% c-x2-bot: commit time of T_x2 (drawn at the bottom)
	\draw let \p{p-x2} = (x2.east), \p{p-s-ti-mid} = (s-ti-mid) in
	  node (c-x2-bot) [cs] at (\x{p-x2}, \y{p-s-ti-mid}) {$c_{x_2}$};
	\draw[dashline] (x2.south east) to (c-x2-bot.north);

	% c-y5-top: commit time of T_y5 (drawn at the top)
	\draw let \p{p-y5} = (y5.east), \p{p-s-ti-mid} = (s-ti-mid) in
	  node (c-y5-top) [cs] at (\x{p-y5}, \y{p-s-ti-mid}) {$c_{y_5}$};
	\draw[dashline] (y5.north east) to (c-y5-top.south);

	% k3 at center of c-x2-bot and c-y5-top
	\node (k3) [knode] at ($(c-x2-bot) !0.5! (c-y5-top)$) {$k_3$};
	\draw [toarrow] (k3) to (c-x2-bot);
	\draw [toarrow] (k3) to (c-y5-top);
    \end{scope}
  }
\end{tikzpicture}
}
  % \scalebox{0.20}{\tikzset{trans-node/.style = {draw, line width = 3pt, inner sep = 12pt}}
\tikzset{read-from/.style = {>={Stealth[length = 12pt]}, ->, dash pattern = on 20pt off 10pt, thick}}
\tikzset{knode/.style = {inner sep = 8pt, outer sep = 5pt, font=\fontsize{40}{40}\selectfont}}
\tikzset{toarrow/.style = {>={Stealth[length = 12pt]}, ->, ultra thick}}
\tikzset{dashline/.style = {ultra thick, dash pattern = on 20pt off 10pt}}
\tikzset{cs/.style = {outer sep = 5pt, font=\fontsize{40}{40}\selectfont}}  % c: commit; s: start

% #1: numbering
\newcommand{\xtrans}[1]{\node (x#1) [trans-node, on chain = x, font=\fontsize{30}{30}\selectfont] {$T_{x_#1}: w_{x_#1}(x_#1)$}}
\newcommand{\ytrans}[1]{\node (y#1) [trans-node, on chain = y, font=\fontsize{30}{30}\selectfont] {$T_{y_#1}: w_{y_#1}(y_#1)$}}

\begin{tikzpicture}[start chain = x going right,
  		start chain = y going right,
	      	node distance = 1.0cm,
	        font = \huge]
  % transactions updating x
  \begin{scope}
    \foreach \i in {1, ..., 6}
      \xtrans{\i};
  \end{scope}

  % transactions updating y
  \begin{scope}
    \node (y1) [trans-node, on chain = y, below left = 3.0cm and -2.0cm of x3, font=\fontsize{30}{30}\selectfont] {$T_{y_1}: w_{y_1}(y_1)$};
    \foreach \i in {2, ..., 6}
      \ytrans{\i};
  \end{scope}
   
  % transaction T_i reading x and y
  \begin{scope}
    \node (ti) [trans-node, below right = 2.0cm and 2.5cm of y2, font=\fontsize{35}{35}\selectfont] 
	  {$T_i: \hskip 3em r_i(x_2) \hskip 8em r_i(y_5) \hskip 4em$};
  \end{scope}

  \uncover<2->{
    % T_i reads x2
    \begin{pgfonlayer}{background}
	  \coordinate (c-rx2) at ($(ti.north) !0.5! (ti.north west)$);
	  \draw[read-from]  (x2.south) to (c-rx2); % [out = -70, in = 120] (c-rx2);
    \end{pgfonlayer}
    
    % T_i reads y5
    \coordinate (c-ry5) at ($(ti.north) !0.5! (ti.north east)$);
    \draw[read-from] (y5.south) to (c-ry5);
  }

  \uncover<3->{
    % k1
    \def\konespace{11cm}
    \begin{scope}[on background layer]
	% c-x2-top: commit time of T_x2
	\draw let \p{p-x2} = (x2.east), \p{p-ti} = (ti.north west) in
	    node (c-x2-top) [cs] at ($(\x{p-x2}, \y{p-ti}) + (0.0cm, \konespace)$) {$c_{x_2}$};
	\draw[dashline] (x2.north east) to (c-x2-top.south);

	% s-ti-top: start time of T_i (drawn at the top)
	\node (s-ti-top) [cs] at ($(ti.north west) + (0.0cm, \konespace)$) {$s_i$};
	\draw[dashline] let \p{p-ti} = (ti.west), \p{p-x4} = (x4.west) in
		  (s-ti-top.south) to (\x{p-ti}, \y{p-x4});

	% k1 at center of c-x2-top and s-ti-top
	\node (k1) [knode] at ($(c-x2-top) !0.5! (s-ti-top)$) {$k_1$};
	\draw [toarrow] (k1) to (c-x2-top);
	\draw [toarrow] (k1) to (s-ti-top);
    \end{scope}
  }

  \uncover<4->{
    % k2
    \def\ktwospace{1.5cm}
    \begin{scope}[on background layer]
	% s-ti-bot: start time of T_i (drawn at the bottom)
	\node (s-ti-bot) [cs] at ($(ti.south west) - (0.0cm, \ktwospace)$) {$s_i$};
	\draw[dashline] let \p{p-ti} = (ti.west), \p{p-x4} = (x4.west) in
	(s-ti-bot.north) to (\x{p-ti}, \y{p-x4});

	% c-y5-bot: commit time of T_y5
	\draw let \p{p-y5} = (y5.east), \p{p-ti} = (ti.south west) in
	node (c-y5-bot) [cs] at ($(\x{p-y5}, \y{p-ti}) - (0.0cm, \ktwospace)$) {$c_{y_5}$};
	\draw [dashline] (y5.south east) to (c-y5-bot.north);

	% k2 at center of s-ti-bot and c-y5-bot
	\node (k2) [knode] at ($(s-ti-bot) !0.5! (c-y5-bot)$) {$k_2$};
	\draw [toarrow] (k2) to (s-ti-bot);
	\draw [toarrow] (k2) to (c-y5-bot);
    \end{scope}
  }

  \uncover<5->{
    % k3
    \begin{scope}[on background layer]
	% (invisible; as positioning anchor) s-ti-mid: start time of T_i (drawn in the middle)
	\node (s-ti-mid) [cs] at ($(s-ti-top) !0.35! (s-ti-bot)$) {};

	% c-x2-bot: commit time of T_x2 (drawn at the bottom)
	\draw let \p{p-x2} = (x2.east), \p{p-s-ti-mid} = (s-ti-mid) in
	  node (c-x2-bot) [cs] at (\x{p-x2}, \y{p-s-ti-mid}) {$c_{x_2}$};
	\draw[dashline] (x2.south east) to (c-x2-bot.north);

	% c-y5-top: commit time of T_y5 (drawn at the top)
	\draw let \p{p-y5} = (y5.east), \p{p-s-ti-mid} = (s-ti-mid) in
	  node (c-y5-top) [cs] at (\x{p-y5}, \y{p-s-ti-mid}) {$c_{y_5}$};
	\draw[dashline] (y5.north east) to (c-y5-top.south);

	% k3 at center of c-x2-bot and c-y5-top
	\node (k3) [knode] at ($(c-x2-bot) !0.5! (c-y5-top)$) {$k_3$};
	\draw [toarrow] (k3) to (c-x2-bot);
	\draw [toarrow] (k3) to (c-y5-top);
    \end{scope}
  }
\end{tikzpicture}
}
  % \tikzset{trans-node/.style = {draw, line width = 3pt, inner sep = 12pt}}
\tikzset{read-from/.style = {>={Stealth[length = 12pt]}, ->, dash pattern = on 20pt off 10pt, thick}}
\tikzset{knode/.style = {inner sep = 8pt, outer sep = 5pt, font=\fontsize{40}{40}\selectfont}}
\tikzset{toarrow/.style = {>={Stealth[length = 12pt]}, ->, ultra thick}}
\tikzset{dashline/.style = {ultra thick, dash pattern = on 20pt off 10pt}}
\tikzset{cs/.style = {outer sep = 5pt, font=\fontsize{40}{40}\selectfont}}  % c: commit; s: start

% #1: numbering
\newcommand{\xtrans}[1]{\node (x#1) [trans-node, on chain = x, font=\fontsize{30}{30}\selectfont] {$T_{x_#1}: w_{x_#1}(x_#1)$}}
\newcommand{\ytrans}[1]{\node (y#1) [trans-node, on chain = y, font=\fontsize{30}{30}\selectfont] {$T_{y_#1}: w_{y_#1}(y_#1)$}}

\begin{tikzpicture}[start chain = x going right,
  		start chain = y going right,
	      	node distance = 1.0cm,
	        font = \huge]
  % transactions updating x
  \begin{scope}
    \foreach \i in {1, ..., 6}
      \xtrans{\i};
  \end{scope}

  % transactions updating y
  \begin{scope}
    \node (y1) [trans-node, on chain = y, below left = 3.0cm and -2.0cm of x3, font=\fontsize{30}{30}\selectfont] {$T_{y_1}: w_{y_1}(y_1)$};
    \foreach \i in {2, ..., 6}
      \ytrans{\i};
  \end{scope}
   
  % transaction T_i reading x and y
  \begin{scope}
    \node (ti) [trans-node, below right = 2.0cm and 2.5cm of y2, font=\fontsize{35}{35}\selectfont] 
	  {$T_i: \hskip 3em r_i(x_2) \hskip 8em r_i(y_5) \hskip 4em$};
  \end{scope}

  \uncover<2->{
    % T_i reads x2
    \begin{pgfonlayer}{background}
	  \coordinate (c-rx2) at ($(ti.north) !0.5! (ti.north west)$);
	  \draw[read-from]  (x2.south) to (c-rx2); % [out = -70, in = 120] (c-rx2);
    \end{pgfonlayer}
    
    % T_i reads y5
    \coordinate (c-ry5) at ($(ti.north) !0.5! (ti.north east)$);
    \draw[read-from] (y5.south) to (c-ry5);
  }

  \uncover<3->{
    % k1
    \def\konespace{11cm}
    \begin{scope}[on background layer]
	% c-x2-top: commit time of T_x2
	\draw let \p{p-x2} = (x2.east), \p{p-ti} = (ti.north west) in
	    node (c-x2-top) [cs] at ($(\x{p-x2}, \y{p-ti}) + (0.0cm, \konespace)$) {$c_{x_2}$};
	\draw[dashline] (x2.north east) to (c-x2-top.south);

	% s-ti-top: start time of T_i (drawn at the top)
	\node (s-ti-top) [cs] at ($(ti.north west) + (0.0cm, \konespace)$) {$s_i$};
	\draw[dashline] let \p{p-ti} = (ti.west), \p{p-x4} = (x4.west) in
		  (s-ti-top.south) to (\x{p-ti}, \y{p-x4});

	% k1 at center of c-x2-top and s-ti-top
	\node (k1) [knode] at ($(c-x2-top) !0.5! (s-ti-top)$) {$k_1$};
	\draw [toarrow] (k1) to (c-x2-top);
	\draw [toarrow] (k1) to (s-ti-top);
    \end{scope}
  }

  \uncover<4->{
    % k2
    \def\ktwospace{1.5cm}
    \begin{scope}[on background layer]
	% s-ti-bot: start time of T_i (drawn at the bottom)
	\node (s-ti-bot) [cs] at ($(ti.south west) - (0.0cm, \ktwospace)$) {$s_i$};
	\draw[dashline] let \p{p-ti} = (ti.west), \p{p-x4} = (x4.west) in
	(s-ti-bot.north) to (\x{p-ti}, \y{p-x4});

	% c-y5-bot: commit time of T_y5
	\draw let \p{p-y5} = (y5.east), \p{p-ti} = (ti.south west) in
	node (c-y5-bot) [cs] at ($(\x{p-y5}, \y{p-ti}) - (0.0cm, \ktwospace)$) {$c_{y_5}$};
	\draw [dashline] (y5.south east) to (c-y5-bot.north);

	% k2 at center of s-ti-bot and c-y5-bot
	\node (k2) [knode] at ($(s-ti-bot) !0.5! (c-y5-bot)$) {$k_2$};
	\draw [toarrow] (k2) to (s-ti-bot);
	\draw [toarrow] (k2) to (c-y5-bot);
    \end{scope}
  }

  \uncover<5->{
    % k3
    \begin{scope}[on background layer]
	% (invisible; as positioning anchor) s-ti-mid: start time of T_i (drawn in the middle)
	\node (s-ti-mid) [cs] at ($(s-ti-top) !0.35! (s-ti-bot)$) {};

	% c-x2-bot: commit time of T_x2 (drawn at the bottom)
	\draw let \p{p-x2} = (x2.east), \p{p-s-ti-mid} = (s-ti-mid) in
	  node (c-x2-bot) [cs] at (\x{p-x2}, \y{p-s-ti-mid}) {$c_{x_2}$};
	\draw[dashline] (x2.south east) to (c-x2-bot.north);

	% c-y5-top: commit time of T_y5 (drawn at the top)
	\draw let \p{p-y5} = (y5.east), \p{p-s-ti-mid} = (s-ti-mid) in
	  node (c-y5-top) [cs] at (\x{p-y5}, \y{p-s-ti-mid}) {$c_{y_5}$};
	\draw[dashline] (y5.north east) to (c-y5-top.south);

	% k3 at center of c-x2-bot and c-y5-top
	\node (k3) [knode] at ($(c-x2-bot) !0.5! (c-y5-top)$) {$k_3$};
	\draw [toarrow] (k3) to (c-x2-bot);
	\draw [toarrow] (k3) to (c-y5-top);
    \end{scope}
  }
\end{tikzpicture}

\end{frame}
%%%%%%%%%%%%%%%

%%%%%%%%%%%%%%%
\begin{frame}{}
  \begin{align*}
    & \red{(\konebv{})} \\
    \forall r_i(x_{j}), w_k(x_k), c_k \in h: 
    & \left(c_j \in h \land \bigwedge_{k=1}^{m} \left(c_j \hp c_k \hp s_i \right)\right) \Rightarrow m < k_1, \\[10pt]
    & \red{(\ktwofv{})} \\
    \forall r_i(x_{j}), w_k(x_k), c_k \in h:
    & \left(c_j \in h \land \bigwedge_{k=1}^{m} \left(s_i \hp c_k \hp c_j \right)\right) \Rightarrow m \le k_2, \\[10pt]
    & \red{(\kthreesv{})} \\
    \forall r_i(x_j), r_i(y_l), w_k(x_k), c_k \in h:
    & \left(\bigwedge_{k=1}^{m} \left(c_j \hp c_k \hp c_l \right)\right) \Rightarrow m \le k_3.
  \end{align*}
\end{frame}
%%%%%%%%%%%%%%%

%%%%%%%%%%%%%%%
\begin{frame}{}
  \begin{equation*}
    h \in \rvsi{} \iff h \in \konebv{} \cap \ktwofv{} \cap \kthreesv{} \cap \wcf{}
  \end{equation*}

  \vspace{0.50cm}
  \[
    \rvsi(\infty,\infty,\infty) = \rc \qquad \rvsi(1,0,\ast) = \si
  \]
\end{frame}
%%%%%%%%%%%%%%%
