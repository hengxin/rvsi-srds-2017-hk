% si
%%%%%%%%%%%%%%%
\begin{frame}{}
  $\text{Transaction } T_i:\quad s_i\quad (r_i/w_i)^{+}\quad c_i/a_i$

  \begin{description}
    \item[$s_i$:] \texttt{start} operation
    \item[$r_i/w_i$:] \texttt{read/write} operation
    \item[$c_i/a_i$:] \texttt{commit/abort} operation
  \end{description}

  \pause
  \vspace{0.20cm}
  \begin{description}[$w_i(x_i)$:]
    \item[$x_i$:] version $i$ of data item $x$ written by $T_i$
    \item[$r_i(x_j)$:] transaction $T_i$ reading $x_j$
    \item[$w_i(x_i)$:] transaction $T_i$ writing $x_i$
  \end{description}

  \pause
  \vspace{0.80cm}
  History: modelling an execution of a transactional key-value store
  \begin{itemize}
    \item \emph{time-precedes partial order} $\timeprech$ over all operations
  \end{itemize}
\end{frame}
%%%%%%%%%%%%%%%

%%%%%%%%%%%%%%%
\begin{frame}{}
  A history $h$ is in \emph{\large snapshot isolation} iff it satisfies \citeinbeamer{Adya}{Thesis}{99} \\[0.20cm]
  \begin{description}[Snapshot Write:]
    \item[Snapshot Read:] % All reads of transaction $T_i$ occur at $T_i$'s start time.
      Each transaction reads data from the ``lastest'' snapshot as of the time it started.
      \begin{align*}
	\forall &r_i(x_{j \neq i}), w_{k \neq j}(x_k), c_k \in h: \\
	& (c_j \in h \land c_j \timeprech s_i)
	 \land (s_i \timeprech c_k \lor c_k \timeprech c_j).
      \end{align*}
    \pause
    \item[Snapshot Write:] No concurrent committed transactions may write the same data item. (WCF: write-conflict freedom)
      \begin{align*}
	\forall w_i(x_i), w_{j \neq i}(x_j) \in h \implies (c_i \timeprech s_j \lor c_j \timeprech s_i).
      \end{align*}
  \end{description}
\end{frame}
%%%%%%%%%%%%%%%
