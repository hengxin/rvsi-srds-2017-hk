\section{Experimental Evaluation}

%%%%%%%%%%%%%%%
\begin{frame}{}
  \begin{center}
    Impacts of \rvsi{} specification \\[5pt]
    on the \emph{transaction abort rates} in various scenarios
  \end{center}

  \pause
  \vspace{0.50cm}
  \centerline{\red{Performance?}}
  \begin{itemize}
    \item Not done yet in this work
    \item \chameleon{} prototype is $\ldots$
  \end{itemize}
\end{frame}
%%%%%%%%%%%%%%%

%%%%%%%%%%%%%%%
\begin{frame}{}
  Transactions abort for two reasons:
  \begin{itemize}
    \item ``vc-aborted'': \rvsi{} version constraints violated
    \item ``wcf-aborted'': the WCF property violated
  \end{itemize}

  \pause
  \vspace{0.60cm}
  Transaction abort rates due to \red{``vc-aborted''} are \emph{sensitive} to different values of $k_1$, $k_2$, or $k_3$,
  \pause
  but those due to \blue{``wcf-aborted''} are not.

  \pause
  \vspace{0.50cm}
  \[
    h \in \rvsi{} \iff h \in \konebv{} \cap \ktwofv{} \cap \kthreesv{} \cap \wcf{}.
  \]
\end{frame}
%%%%%%%%%%%%%%%

%%%%%%%%%%%%%%%
\begin{frame}{}
  \begin{columns}
    \column{0.50\textwidth}
      \chameleon{} prototype on Aliyun:
      \begin{itemize}
	\item 3 datacenters~\footnotemark[1]
	\item 3 nodes in each datacenter
	\item Partition \& Replication
	\item Clients in our lab~\footnotemark[2]
      \end{itemize}

      \vspace{0.60cm}
      \uncover<2->{
	(One-way) delays among nodes~\footnotemark[3]:
	\begin{description}[Across datacenters:]
	  \item[Within datacenter:] $1 \sim 2$ms
	  \item[Across datacenters:] $15 \sim 25$ms
	  \item[Clients to nodes:] $15 \sim 20$ms
	\end{description}
      }
    \column{0.50\textwidth}
      \fignocaption{width = 0.85\textwidth}{figs/chameleon-arch.pdf}
  \end{columns}

  \footnotetext[1]{Located in East China, North China, and South China, respectively.}
  \footnotetext[2]{Located in East China.}
  \footnotetext[3]{\url{https://github.com/hengxin/aliyun-ping-traces}}
\end{frame}
%%%%%%%%%%%%%%%

%%%%%%%%%%%%%%%
\begin{frame}{}
  \begin{table}[!t]
  \centering
  \renewcommand*{\arraystretch}{1.2}
  \caption{Three categories of workload parameters for experiments on Aliyun.}
  % \resizebox{0.95\textwidth}{!}{%
    { %
  \begin{tabular}{|c|c||c|c|}
	\hline
    \multicolumn{2}{|c||}{\textbf{Parameter}} & \textbf{Value}		& \textbf{Explanation}
	\\ \hline  \hline
    \multirow{6}{*}{\bf Transaction-related}
    &\#keys  						& 25 = 5 (rows) $\times$ 5 (columns)  	&  	size of keyspace
	\\ \cline{2-4}
	&\#clients						& 5, 10, 15, 20, 25, 30 & number of clients
    \\ \cline{2-4}
	&\#txs/client					& 1000 & number of transactions per client
	\\ \cline{2-4}
	&\#ops/tx				& $\sim$ Binomial(20, 0.5) &  number of operations per transaction
	\\ \cline{2-4}
	&rwRatio							& 1:2, 1:1, 4:1 & {\#reads}/{\#writes}
	\\ \cline{2-4}
	&zipfExponent					& 1		& parameter for Zipfian distribution
	\\ \hline  \hline
    \multirow{3}{*}{\bf Execution-related} & minInterval						& 0ms		& minimum inter-transaction time
	\\ \cline{2-4}
	&maxInterval						& 10ms		& maximum inter-transaction time
	\\ \cline{2-4}
	&meanInterval					& 5ms		& mean inter-transaction time
    \\ \hline \hline
    {\bf RVSI-related} & $(k_1, k_2, k_3)$		
		&  \innercell{c}{(1,0,0) (1,1,0) (1,1,1) \\ (2,0,0) (2,0,1) (2,1,1)}	
		&  for $\konebv{}$, $\ktwofv{}$, and $\kthreesv{}$
	\\ \hline
  \end{tabular}}
  % }
\end{table}

\end{frame}
%%%%%%%%%%%%%%%

%%%%%%%%%%%%%%%
\begin{frame}{}
  under read-frequent workloads
\end{frame}
%%%%%%%%%%%%%%%

%%%%%%%%%%%%%%%
\begin{frame}{}
  \fignocaption{width = 0.80\textwidth}{figs/aliyun-vc-rf.pdf}

  \pause
  The transaction abort rates due to ``vc-aborted'' \pause can be \red{greatly reduced}
  by \blue{slightly} increasing the values of $k_1$, $k_2$, or $k_3$:
  \[
    vc(1,0,0) = 0.1994 \implies vc(2,1,1) = 0.0091 \quad (\text{\#clients} = 30)
  \]
\end{frame}
%%%%%%%%%%%%%%%

%%%%%%%%%%%%%%%
\begin{frame}{}
  \fignocaption{width = 0.55\textwidth}{figs/aliyun-bvfvsv.pdf}

  \pause
  Most ``vc-aborted'' transactions abort because of violating \red{$\ktwofv$}.
  \[
    fv(1,0,0) = 0.1889 \implies fv(\blue{2},0,0) = 0.1866 \implies fv(1,\red{\bf 1},0) = 0.0064
  \]
\end{frame}
%%%%%%%%%%%%%%%

%%%%%%%%%%%%%%%
\begin{frame}{}
  \begin{center}
    \red{Question: when does $k_1$ for $\konebv$ take effect?} \\[10pt]
    It seems that $\konebv$ has \emph{little} impact on the transaction abort rates. \\[30pt]
    \pause
    It may be the case in the Aliyun scenarios. \\[10pt]
    \blue{What about other scenarios?}
  \end{center}
\end{frame}
%%%%%%%%%%%%%%%

%%%%%%%%%%%%%%%
\begin{frame}{}
  \centerline{Three types of delays for \blue{\Large controlled experiments} on local hosts.}
  \begin{table}[!t]
  \centering
  \renewcommand*{\arraystretch}{1.2}
  \begin{tabular}{|c||c|c|}
	\hline
	{\bf Types}		& {\bf Values (ms)}	& {\bf Explanation}
	\\ \hline \hline
	{\bf issueDelay}	& 5, 8, 10, 12, 15, 20	& delays between clients and replicas
	\\ \hline
	{\bf replDelay}	& 5, 10, 15, 20, 30 	& delays between masters and slaves
	\\ \hline
	{\bf 2pcDelay}	& 10, 20, 30, 40, 50 	& delays among masters
	\\ \hline
  \end{tabular}
\end{table}

\end{frame}
%%%%%%%%%%%%%%%

%%%%%%%%%%%%%%%
\begin{frame}{}
  \fignocaption{width = 0.50\textwidth}{figs/simlog-rw05.pdf}

  \begin{center}
    When the \blue{``issueDelay''} gets shorter, \\
    the impacts of $\ktwofv$ go weaker, \\
    and the impacts of $\konebv$ have begun to emerge.
  \end{center}
\end{frame}
%%%%%%%%%%%%%%%

%%%%%%%%%%%%%%%
\begin{frame}{}
  \begin{description}[issueDelay = 20ms:]
    \setlength{\itemsep}{10pt}
    \item[issueDelay = 20ms:] $bv(1,0,0) = \red{0.0057} \quad\;\; fv(1,0,0) = 0.0251$
    \item[issueDelay = 15ms:] $bv(1,0,0) = \red{0.08225} \quad fv(1,0,0) = 0.0393$
    \item[issueDelay = 5ms:] $bv(1,0,0) = \red{0.1716} \quad\;\; fv(1,0,0) = 0.0045$
  \end{description}

  \pause
  \vspace{0.50cm}
  \begin{center}
    \blue{larger issueDelay} $\implies$ longer transaction \\[5pt]
    more concurrent transactions \\[5pt]
    more likely to obtain data versions updated by concurrent transactions \\[5pt]
    more sensitive to \blue{$\ktwofv$}
  \end{center}
\end{frame}
%%%%%%%%%%%%%%%

%%%%%%%%%%%%%%%
\begin{frame}{}
  Generally, \rvsi{} \blue{helps to reduce} the transaction abort rates 
  when applications are willing to tolerate certain anomalies.

  \pause
  \vspace{0.6cm}
  \begin{description}[<+->]
    \setlength{\itemsep}{5pt}
    \item[$\ktwofv$:] In the Aliyun scenarios, most transactions have been aborted because of violating $\ktwofv$.
    \item[$\konebv$:] In controlled experiments, the impacts of $\konebv$ emerge when the issueDelay gets shorter.
    \item[$\kthreesv$:] \uncover<4->{\textcolor{gray}{Complex and challenging (involving multiple data items)}}
  \end{description}
\end{frame}
%%%%%%%%%%%%%%%
