\appendix

%%%%%%%%%%%%%%%
\begin{frame}{}
  \begin{columns}
    \column{0.50\textwidth}
      Two transactions are \emph{concurrent} if
      \[
	s_i \timeprech c_j \land s_j \timeprech c_i
      \]
    \column{0.50\textwidth}
      \fignocaption{width = 0.60\textwidth}{figs/concur-tx}
  \end{columns}
\end{frame}
%%%%%%%%%%%%%%%

%%%%%%%%%%%%%%%
\begin{frame}{}
  \resizebox{1.00\textwidth}{!}{\input{tikz/rvsi-def}}
\end{frame}
%%%%%%%%%%%%%%%

%%%%%%%%%%%%%%%
\begin{frame}{}
  \begin{center}
    An online bookstore application~\footnote{Adapted from \citeinbeamer{Guo}{SIGMOD}{04} 
    and \citeinbeamer{Bernstein}{SIGMOD}{06}.} for motivating \\
    \red{\large ``bounded inconsistency''} and \red{\large ``runtime-tuable''}:
  \end{center}

  \vspace{-0.20cm}
  \begin{table}
    \centering
    \begin{tabular}{c|c|c|c|c|c|c}
      \hline
      Title & Authors & Sales & Inventory & Ratings & Reviews & $\cdots$ \\
      \hline
    \end{tabular}
  \end{table}

  \vspace{0.20cm}
  \begin{description}[Bookstore Clerk ($T_2$):]
    \setlength{\itemsep}{6pt}
    \item[Customer ($T_1$):] Obtaining the basic info. about a book
      \begin{itemize}
	\item \emph{\blue{out-of-date}} reviews
      \end{itemize}
      \pause
    \item[Bookstore Clerk ($T_2$):] Checking the inventory of a book
      \begin{itemize}
	\item updated by concurrent transactions that commit \emph{\blue{after}} $T_2$ starts
      \end{itemize}
      \pause
    \item[Sales Analyst ($T_3$):] Studying sales vs. ratings of a book
      \begin{itemize}
	\item sales and ratings from \emph{\blue{separate snapshots}}
      \end{itemize}
  \end{description}
\end{frame}
%%%%%%%%%%%%%%%

%%%%%%%%%%%%%%%
\begin{frame}{Applicability}
\end{frame}
%%%%%%%%%%%%%%%

%%%%%%%%%%%%%%%
\begin{frame}{}
  In terms of \emph{event} generation and handling:
  \begin{description}
    \item[Clients:] \ebegin, \eread, \ewrite, \eend%
    \item[Master:] \estart, \ecommit, \esend%
    \item[Slaves:] \ereceive%
  \end{description}
\end{frame}
%%%%%%%%%%%%%%%

%%%%%%%%%%%%%%%
\begin{frame}{}
  \begin{center}
    \begin{minipage}{1.0\textwidth}
      %%%%%%%%%%%%%%%%%%%%%%%%%%%%%%%%%%%%%%%% For clients %%%%%%%%%%%%%%%%%%%%%%%%%%%%%%%%%%%%%%%%
\setcounter{algorithm}{0}
\begin{algorithm}[H]
  \caption{\rvsims{} Protocol for Executing Transaction $T$ \red{(Client)}.}
  \begin{algorithmic}[1]
    \Procedure{begin}{\null}
    \State $T.\attr{sts}$ $\gets$ \algkeyword{rpc-call} \Call{start}{\null} at master 
    $\master$   \label{line:call-start}
    \EndProcedure

    \Procedure{read}{$x$}
      \State $x.\attr{ver}$ $\gets$ \algkeyword{rpc-call} \Call{read}{$x$} at any site
    \EndProcedure

    \Procedure{write}{$x, v$}
      \State add $(x, v)$ to $T.\attr{writes}$ \label{line:write-at-client}
    \EndProcedure

    \Procedure{end}{$T$}
    \State \red{$T.\attr{vc}$ $\gets$ \Call{add-vc}{\null}} \label{line:call-add-vc}
    \State $c/a$ $\gets$ \algkeyword{rpc-call} \Call{commit}{$T.\attr{writes}, 
    T.\attr{vc}$} at $\master$ \label{line:call-commit}
    \EndProcedure
  \end{algorithmic}
\end{algorithm}
%%%%%%%%%%%%%%%%%%%%%%%%%%%%%%%%%%%%%%%% For clients %%%%%%%%%%%%%%%%%%%%%%%%%%%%%%%%%%%%%%%%

    \end{minipage}
  \end{center}
\end{frame}
%%%%%%%%%%%%%%%

%%%%%%%%%%%%%%%
\begin{frame}{}
  \begin{center}
    \scalebox{0.85}{
      \begin{minipage}{1.0\textwidth}
	%%%%%%%%%%%%%%%%%%%%%%%%%%%%%%%%%%%%%%%% For master %%%%%%%%%%%%%%%%%%%%%%%%%%%%%%%%%%%%%%%%
\setcounter{algorithm}{0}
\begin{algorithm}[H]
  \caption{\rvsims{} Protocol for Executing Transaction $T$ \red{(Master)}.}
  \begin{algorithmic}[1]
    \Statex $\master$.\attr{ts}: for start-timestamps and commit-timestamps
    \Statex $\set{x.\attr{ver} = (x.\attr{ts}, x.\attr{ord}, x.\attr{val})}$: set of versions of $x$
    \hStatex

    \Procedure{start}{\null}
    \State \Return ++$\master.\attr{ts}$
    \EndProcedure
    
    \Procedure{read}{$x$}
      \State \Return the latest $x.\attr{ver}$ installed \label{line:read-at-master}
    \EndProcedure

    \Procedure{commit}{$T.\attr{writes}, T.\attr{vc}$}
    \If{\red{\Call{check-vc}{$T.\attr{vc}$}} \&\& write-conflict freedom}   
    \label{line:call-check-vc}
      \State $T.\attr{cts}$ $\gets$ ++$\master.\attr{ts}$ 

      \lComment{apply $T.\attr{writes}$ locally and propagate it} 
      \State $T.\textrm{\it upvers} = \emptyset$  \mComment{collect updated versions}  
      \label{line:commit-updates}
      \For{$(x,v) \in T.\attr{writes}$}    
      \State $x.\attr{new-ver} \gets (T.\attr{cts}, \textrm{++}x.\attr{ord}, v)$
      \State add $x.\attr{new-ver}$ to $\set{x.\attr{ver}}$ and $T.\textrm{\it upvers}$
      \EndFor 
      \State \algkeyword{broadcast} \msg{PROP}{T.\textrm{\it upvers}} to slaves 
      \label{line:commit-prop}
      % \lendComment{apply $T.\attr{writes}$ locally and propagate it} 

      \State \Return $c$ denoting ``committed''
    \EndIf
    \State \Return $a$ denoting ``aborted''
    \EndProcedure
  \end{algorithmic}
\end{algorithm}
%%%%%%%%%%%%%%%%%%%%%%%%%%%%%%%%%%%%%%%% For master %%%%%%%%%%%%%%%%%%%%%%%%%%%%%%%%%%%%%%%%

      \end{minipage}
    }
  \end{center}
\end{frame}
%%%%%%%%%%%%%%%

%%%%%%%%%%%%%%%
\begin{frame}{}
  \begin{center}
    \begin{minipage}{1.0\textwidth}
      %%%%%%%%%%%%%%%%%%%%%%%%%%%%%%%%%%%%%%%% For slaves %%%%%%%%%%%%%%%%%%%%%%%%%%%%%%%%%%%%%%%%
\setcounter{algorithm}{0}
\begin{algorithm}[H]
  \caption{\rvsims{} Protocol for Executing Transaction $T$ \red{(Slave)}.}
  \begin{algorithmic}[1]
    \Statex $x.\attr{ver} = (x.\attr{ts}, x.\attr{ord}, x.\attr{val})$: the latest version of $x$
    \hStatex

    \Procedure{read}{$x$}
    \State \Return $x.\attr{ver}$
    \EndProcedure

    \algrenewcommand\algorithmicprocedure{\textbf{upon}}
    \Procedure{received}{\msg{PROP}{T.\textrm{\it upvers}}} \label{line:received}
    \For{$\left(x.\attr{ver}' = (x.\attr{ts}', x.\attr{ord}', x.\attr{val}')\right) \in 
    T.\textrm{\it upvers}$}
      \If{$x.\attr{ord}' > x.\attr{ord}$}  
      \State $x.\attr{ver} \gets x.\attr{ver}'$
      \EndIf
    \EndFor
    \EndProcedure
  \end{algorithmic}
\end{algorithm}
%%%%%%%%%%%%%%%%%%%%%%%%%%%%%%%%%%%%%%%% For slaves %%%%%%%%%%%%%%%%%%%%%%%%%%%%%%%%%%%%%%%%

    \end{minipage}
  \end{center}
\end{frame}
%%%%%%%%%%%%%%%

%%%%%%%%%%%%%%%
\begin{frame}{}
  \begin{center}
    \begin{minipage}{1.0\textwidth}
      %%%%%%%%%%%%%%%%%%%%%%%%%%%%%%%%%%%%%%%% For clients %%%%%%%%%%%%%%%%%%%%%%%%%%%%%%%%%%%%%%%%
\setcounter{algorithm}{1}
\begin{algorithm}[H]
  \caption{\rvsimp{} for Executing Transaction $T$ \red{(Client)}.}
  \begin{algorithmic}[1]
    \Procedure{begin}{\null}
	\State \Return \algkeyword{rpc-call} \Call{getTS}{\null} at $\timeoracle{}$
		\label{line:rvsimp-client-call-getts}
    \EndProcedure

    \Procedure{end}{\null}	\label{line:rvsimp-call-end}
      \State $T.\attr{vc}$ $\gets$ \red{\Call{add-vc}{\null}} \label{line:rvsimp-call-add-vc}
      \State $c/a$ $\gets$ \algkeyword{rpc-call} \Call{c-commit}{$T.\attr{writes}, T.\attr{vc}$} 
      at $\coordinator$ \label{line:rvsimp-call-commit}
    \EndProcedure
  \end{algorithmic}
\end{algorithm}
%%%%%%%%%%%%%%%%%%%%%%%%%%%%%%%%%%%%%%%% For clients %%%%%%%%%%%%%%%%%%%%%%%%%%%%%%%%%%%%%%%%

    \end{minipage}
  \end{center}
\end{frame}
%%%%%%%%%%%%%%%

%%%%%%%%%%%%%%%
\begin{frame}{}
  \begin{center}
    \begin{minipage}{1.0\textwidth}
      %%%%%%%%%%%%%%%%%%%% For Timestamp Oracle %%%%%%%%%%%%%%%%%%%%
\setcounter{algorithm}{1}
\begin{algorithm}[H]
  \caption{\rvsimp{} for Executing Transaction $T$ \red{(Timestamp Oracle)}.}
  \begin{algorithmic}[1]
    \Statex \tots{}: for start-timestamps and commit-timestamps

    \Procedure{getTS}{\null}	\label{line:rvsimp-getts}
      \State \Return ++\tots{}
    \EndProcedure
  \end{algorithmic}
\end{algorithm}
%%%%%%%%%%%%%%%%%%%% For Timestamp Oracle %%%%%%%%%%%%%%%%%%%%

    \end{minipage}
  \end{center}
\end{frame}
%%%%%%%%%%%%%%%

%%%%%%%%%%%%%%%
\begin{frame}{}
  \begin{center}
    \scalebox{0.90}{
      \begin{minipage}{1.0\textwidth}
	%%%%%%%%%%%%%%%%%%%%%%%%%%%%%%%%%%%%%%%% For coordinator %%%%%%%%%%%%%%%%%%%%%%%%%%%%%%%%%%%%%%%%
\setcounter{algorithm}{1}
\begin{algorithm}[H]
  \caption{\rvsimp{} for Executing Transaction $T$ \red{(Coordinator)}.}
  \begin{algorithmic}[1]
      \Procedure{c-commit}{$T.\attr{writes}, T.\attr{vc}$}	\label{line:rvsimp-ccommit}
	\State \red{split $T.\attr{writes}$ and $\tvc{T}$ with the data partitioning strategy}
	  \label{line:rvsimp-partition}
	  \hStatex

      \lComment{the prepare phase:}
	  \State \algkeyword{rpc-call} \Call{prepare}{$T.\attr{writes}, T.\attr{vc}$} at each $\master$
		\label{line:rvsimp-call-prepare}

      \lComment{the commit phase:}
      \If{all \Call{prepare}{$T.\attr{writes}, T.\attr{vc}$} return \textsl{true}}
		\label{line:rvsimp-prepare-all-true}
	  \State $T.\attr{cts} \gets$ \algkeyword{rpc-call} 
		\Call{getTS}{\null} at $\timeoracle$
		  \label{line:rvsimp-coord-call-getts}
		\State \algkeyword{rpc-call} \Call{commit}{$T.\attr{cts}, T.\attr{writes}$} at each $\master$
		  \label{line:rvsimp-coord-call-commit}
		%   \lComment{early commit notification~\cite{binnig:vldb14}}
        % \State \Return $c$ denoting ``commited''    
      \Else
        \State \algkeyword{rpc-call} \Call{abort}{\null} at each $\master$
		  \label{line:rvsimp-coord-call-abort}
        \State \Return $a$ denoting ``aborted''
      \EndIf

	  \If{all \Call{commit}{$T.\attr{cts}, T.\attr{writes}$} return \textsl{true}}
        \State \Return $c$ denoting ``committed''
	  \Else
		\State \Return $a$ denoting ``aborted''
      \EndIf
    \EndProcedure
  \end{algorithmic}
\end{algorithm}
%%%%%%%%%%%%%%%%%%%%%%%%%%%%%%%%%%%%%%%% For coordinator %%%%%%%%%%%%%%%%%%%%%%%%%%%%%%%%%%%%%%%%

      \end{minipage}
    }
  \end{center}
\end{frame}
%%%%%%%%%%%%%%%

%%%%%%%%%%%%%%%
\begin{frame}{}
  \begin{center}
    \begin{minipage}{1.0\textwidth}
      %%%%%%%%%%%%%%%%%%%%%%%%%%%%%%%%%%%%%%%% For masters %%%%%%%%%%%%%%%%%%%%%%%%%%%%%%%%%%%%%%%%
\setcounter{algorithm}{1}
\begin{algorithm}[H]
  \caption{\rvsimp{} for Executing Transaction $T$ \red{(Master)}.}
  \begin{algorithmic}[1]
    \Procedure{prepare}{$T.\attr{writes}, T.\attr{vc}$} \label{line:rvsimp-prepare}
      \State \Return \red{\Call{check-vc}{$T.\attr{vc}$}} \&\& write-conflict freedom
		\label{line:rvsimp-check-in-prepare}
    \EndProcedure

	\Procedure{commit}{$T.\attr{cts}, T.\attr{writes}$}	\label{line:rvsimp-commit}
	  \lComment{apply $T.\attr{writes}$ locally and propagate it} 
		\label{line:rvsimp-apply-in-commit}
    \EndProcedure

	\Procedure{abort}{\null}  \label{line:rvsimp-abort}
	\lComment{abort}
    \EndProcedure
  \end{algorithmic}
\end{algorithm}
%%%%%%%%%%%%%%%%%%%%%%%%%%%%%%%%%%%%%%%% For masters %%%%%%%%%%%%%%%%%%%%%%%%%%%%%%%%%%%%%%%%

    \end{minipage}
  \end{center}
\end{frame}
%%%%%%%%%%%%%%%

%%%%%%%%%%%%%%%
\begin{frame}{}
  Atomicity of the commit-timestamps:

  \fignocaption{width = 0.70\textwidth}{figs/rvsi-atomic-cts.pdf}
\end{frame}
%%%%%%%%%%%%%%%

%%%%%%%%%%%%%%%
\begin{frame}{Delays}
  (One-way) delays among nodes~\footnotemark:
  \begin{description}[Across datacenters:]
    \item[Within datacenter:] $1 \sim 2$ms
    \item[Across datacenters:] $15 \sim 25$ms
    \item[Clients to nodes:] $15 \sim 20$ms
  \end{description}

  \footnotetext{\url{https://github.com/hengxin/aliyun-ping-traces}}
\end{frame}
%%%%%%%%%%%%%%%

%%%%%%%%%%%%%%%
\begin{frame}{Benchmarks}
  \begin{itemize}
    \item The TPC-C benchmark is commonly used to benchmark relational databases.
    \item The YCSB benchmark \citeinbeamer{Cooper}{SoCC}{10} for distributed key-value stores
      does not support transactions.
  \end{itemize}

  \vspace{0.60cm}
  \centerline{We design our own workloads.}
\end{frame}
%%%%%%%%%%%%%%%

%%%%%%%%%%%%%%%
\begin{frame}{}
  \fignocaption{width = 0.80\textwidth}{figs/aliyun-vcwcf-allinone.pdf}
\end{frame}
%%%%%%%%%%%%%%%

%%%%%%%%%%%%%%%
\begin{frame}{}
  \begin{description}[issueDelay = 20ms:]
    \setlength{\itemsep}{10pt}
    \item[issueDelay = 20ms:] $bv(1,0,0) = \red{0.0057} \quad\;\; fv(1,0,0) = 0.0251$
    \item[issueDelay = 15ms:] $bv(1,0,0) = \red{0.08225} \quad fv(1,0,0) = 0.0393$
    \item[issueDelay = 5ms:] $bv(1,0,0) = \red{0.1716} \quad\;\; fv(1,0,0) = 0.0045$
  \end{description}

  \pause
  \vspace{0.50cm}
  \begin{center}
    \blue{larger issueDelay} $\implies$ longer transaction \\[5pt]
    more concurrent transactions \\[5pt]
    more likely to obtain data versions updated by concurrent transactions \\[5pt]
    more sensitive to \blue{$\ktwofv$}
  \end{center}
\end{frame}
%%%%%%%%%%%%%%%

%%%%%%%%%%%%%%%
%%%%%%%%%%%%%%%
