\usepackage{xeCJK}
% \usepackage{zhnumber}	% counters in Chinese
\usepackage{fontspec}
\usepackage{comment}
\usepackage{xifthen}
\usepackage{verbatim}

\usetheme{CambridgeUS} % try Madrid
\usecolortheme{beaver} % try beaver, dolphin, seahorse
\usefonttheme[onlymath]{serif} % try "professionalfonts"
% \setCJKmainfont{SimSun} % try SimSun

\usepackage{amsmath, amsfonts, amssymb, mathtools, pifont}
\newcommand{\cmark}{\ding{51}}%
\newcommand{\xmark}{\ding{55}}%
\def\checkmark{\tikz\fill[scale=0.5](0,.35) -- (.25,0) -- (1,.7) -- (.25,.15) -- cycle;} 

\usepackage{graphicx, subcaption}
\usepackage[export]{adjustbox}

\usepackage[framemethod=TikZ]{mdframed}

\usepackage[normalem]{ulem} % strike through text
\newcommand{\soutthick}[1]{%
    \renewcommand{\ULthickness}{2.0pt}%
       \sout{#1}%
    \renewcommand{\ULthickness}{.4pt}% Resetting to ulem default
}

\setbeamersize{text margin left = 2em, text margin right = 1em}
\setbeamercolor{footnote mark}{fg = teal}
\setbeamertemplate{itemize items}[default]
\setbeamertemplate{enumerate items}[default]

\usepackage{tikz}
\usepackage{tikz-3dplot}
\usepackage{pgfplots}
\usetikzlibrary{arrows.meta, shapes, positioning, calc, backgrounds, fit}

% for algorithms
\usepackage{algorithm}
\usepackage[noend]{algpseudocode} % also loading algorithmicx
\ExplSyntaxOn
\bool_new:N \l__xeCJK_listings_letter_bool % workaround
\ExplSyntaxOff
\usepackage{listings}
\usepackage{textcomp}

\newcommand{\hStatex}[0]{\vspace{5pt}}
\newcommand{\algvariable}[1]{\texttt{#1}}
\newcommand{\algkeyword}[1]{\textbf{#1}}
\newcommand{\algprocedure}[1]{\textsc{#1}}

% lComment: line comment; mComment: margin comment
\algnewcommand{\lComment}[1]{\State \emph{$\triangleright$ #1}}
\algnewcommand{\lendComment}[1]{\State \textcolor{violet}{\emph{$\triangleleft$ #1}}}
\algnewcommand{\mComment}[1]{\Comment{\emph{#1}}}

\algblockdefx{BeginRepeat}{EndRepeat}{\textbf{repeat}}{}
\algnotext{EndRepeat}

\newcommand{\attr}[1]{\textsl{#1}}
\newcommand{\msg}[2]{$\langle \textup{\scriptsize #1}, #2 \rangle$}

% for tables
\usepackage{multirow}
\newcommand{\innercell}[2]{\begin{tabular}{@{}#1@{}}#2\end{tabular}}
\usepackage{hhline}
%%%%%%%%%%%%%% for appendix %%%%%%%%%%%%%%%%
% http://www-ljk.imag.fr/membres/Jerome.Lelong/latex/appendixnumberbeamer.sty
% Reference: http://tex.stackexchange.com/questions/2541/beamer-frame-numbering-in-appendix
\usepackage{appendixnumberbeamer}
% Add total frame count to slides, optional. From Stefan,
% http://www.latex-community.org/forum/viewtopic.php?f=4&t=2173
\expandafter\def\expandafter\insertshorttitle\expandafter{%
  \insertshorttitle\hfill\insertframenumber\,/\,\inserttotalframenumber}
%%%%%%%%%%%%%% for appendix %%%%%%%%%%%%%%%%

% for fig without caption: #1: width/size; #2: fig file
\newcommand{\fignocaption}[2]{
  \begin{figure}[htp]
    \centering
    \includegraphics[#1]{#2}
  \end{figure}
}

% for fig with caption: #1: width/size; #2: fig file; #3: fig caption
\newcommand{\fig}[3]{
  \begin{figure}[htp]
    \centering
      \includegraphics[#1]{#2}
      \caption{#3}
  \end{figure}
}

% for cite: #1: author; #2: conference #3: year
\newcommand{\citeinbeamer}[3]{{\scriptsize{\textcolor{blue}{[#1@#2'#3]}}}}

\usepackage[backend=bibtex]{biblatex}
\addbibresource{phd-defense-report.bib}

\newcommand{\set}[1]{\{#1\}}
\newcommand{\question}[1]{\textcolor{red}{\centerline{#1}}}
\newcommand{\answer}[1]{\textcolor{blue}{\centerline {#1}}}
\newcommand{\alertred}[1]{\textcolor{red}{#1}}
\newcommand{\alertblue}[1]{\textcolor{blue}{#1}}
\newcommand{\todo}[1]{\textcolor{red}{\textbf{TODO:} #1}}
\newcommand{\mathbfblue}[1]{\textcolor{blue}{$\mathbf{#1}$}}

\newcommand{\papertitle}{Parameterized and Runtime-tunable Snapshot Isolation in Distributed Transactional Key-value Stores}
\newcommand{\shortpapertitle}{Parameterized and Runtime-tunable Snapshot Isolation}

\newcommand{\putop}{\texttt{put(K key, V val)}}
\newcommand{\getop}{\texttt{get(K key)}}

\newcommand{\red}[1]{\textcolor{red}{#1}}
\newcommand{\blue}[1]{\textcolor{blue}{#1}}

\newcommand{\hp}{\prec_{h}}
\newcommand{\aca}{\text{ACA}}
\newcommand{\wcf}{\text{WCF}}
\newcommand{\rc}{\text{RC}}
\newcommand{\si}{\text{SI}}
\newcommand{\rvsi}{\text{RVSI}}
\newcommand{\konebv}{k_1\text{-BV}}
\newcommand{\ktwofv}{k_2\text{-FV}}
\newcommand{\kthreesv}{k_3\text{-SV}}
\newcommand{\chameleon}{\textsc{Chameleon}}

\newcommand{\algnamefont}[1]{#1}
\newcommand{\rvsims}{\algnamefont{RVSI-MS}}
\newcommand{\rvsimp}{\algnamefont{RVSI-MP}}
\newcommand{\ebegin}{\textsc{Begin}}
\newcommand{\eread}{\textsc{Read}}
\newcommand{\ewrite}{\textsc{Write}}
\newcommand{\eend}{\textsc{End}}
\newcommand{\estart}{\textsc{Start}}
\newcommand{\ecommit}{\textsc{Commit}}
\newcommand{\esend}{\textsc{Send}}
\newcommand{\ereceive}{\textsc{Receive}}

\newcommand{\master}{\mathcal{M}}
\newcommand{\slave}{\mathcal{S}}
\newcommand{\coordinator}{\mathcal{C}}
\newcommand{\timeoracle}{\mathcal{T}}
\newcommand{\tots}{$\timeoracle$.\attr{ts}}
\newcommand{\tvc}[1]{#1.\attr{vc}}
\newcommand{\tws}[1]{#1.\attr{writes}}
\newcommand{\tcts}[1]{#1.\attr{cts}}
\newcommand{\mpord}[1]{\mathcal{O}_{#1}}
